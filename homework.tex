% !TEX program=luatex
\documentclass[12pt,a4paper]{article}
\usepackage{luatexja-fontspec}
\setmainjfont{FandolSong}
\usepackage{amsmath}
\usepackage{amsthm}
\usepackage{amssymb}
\usepackage{amsfonts}
\usepackage{enumerate}
\usepackage[colorlinks]{hyperref}
\usepackage{tikz-cd}
\usepackage{geometry}
\geometry{left=2cm,right=1cm, top=3cm,bottom=2cm}
\usepackage{fancyhdr}
\usepackage{fourier-orns}
\renewcommand\headrule{\hrulefill \raisebox{-2.1pt}[10pt][10pt]{\quad\decofourleft\decotwo\decofourright\quad}\hrulefill}

\pagestyle{fancy}
\lhead{邹海涛} 
\rhead{17210180015}
\theoremstyle{definition}
\newtheorem{secdefn}{Definition}[subsection]
\newtheorem{exer}{Exercies}[section]
\renewcommand{\qed}{\hfill\ensuremath{\clubsuit}}
\newcommand*{\qeds}{\hfill\ensuremath{\clubsuit}}
\DeclareMathOperator{\spec}{Spec}
\DeclareMathOperator{\proj}{Proj}
\DeclareMathOperator{\pgl}{PGL}
\DeclareMathOperator{\aut}{Aut}
\DeclareMathOperator{\p}{\mathbb{P}}
\DeclareMathOperator{\A}{\mathbb{A}}
\DeclareMathOperator{\im}{Im}
\DeclareMathOperator{\Char}{char}
\DeclareMathOperator{\ord}{ord}
\begin{document}
\noindent
{\LARGE\underline{\textbf{Algebraic Geometry}}}\\
{\hfill\large  \underline{\textbf{邹海涛}} \\
	\hfill ID: 17210180015}\\
\section{Week 1}
\begin{exer}
	Any nonempty open subset of an irreducible topological space is dense and irreducible.
\end{exer}
Let $X$ be an irreducible space and $U \hookrightarrow X$ be an nonempty open subset of $X$. Let $V_1= X\backslash U $ and $V_2= \bar{U}$. Then we have
\[
V_1 \cup V_2 \supseteq (X \backslash U) \cup U =X
\]
Since $X$ is irreducible and $V_1, V_2$ are closed subsets, we have $V_1 = \emptyset$ or $V_2 = X$. That means $\bar{U} =X$. Hence $U$ is dense open subset of $X$.
We can further prove that $X$ is irreducible if any nonempty open subset of $X$ is dense. Otherwise, $X$ is reducible, then $X = V_1 \cup V_2$ where $V_1$ and $V_2$ are non-trivial closed subset of $X$. Then $(X\backslash V_1) \cap (X\backslash V_2) = \emptyset$. It implies $X\backslash V_1$ is non-empty open subset of $X$ which is not dense. Hence we can conclude that any nonempty open subset of irreducible space $X$ is irreducible because its open subsets are all dense in $X$, also in itself.
\qeds
\begin{exer}
	Let $Y$ be an affine variety of dimension $r$ in $\mathbb{A}^n$. Let $H$ be a hypersurface in $\mathbb{A}^n$ and assume that $Y \subsetneq H$. Then every irreducible component of $Y \cap H$ has dimension $r-1$.
\end{exer}
Suppose $H$ be irreducible. It means ideal of $H$ is prime ideal $(f)$ of $k[x_1, \cdots, x_n]$. Let $Y \cap H = V_1 \cup \cdots \cup V_k$ be the irreducible components decomposition and ideal of $V_i$ is $\mathfrak{p}_i$. Hence we have
\[
I(V \cap H) = \mathfrak{p}_1 \cap \cdots \cap \mathfrak{p}_k
\]
Since $Y \varsubsetneq H$, we have $f \notin I(Y)$. Hence the minimal prime ideals of $I(Y) + (f)$ is with height $n-k+1$ by Krull principal ideal theorem, since $I(Y)$ is of height $n-r$. We claim that $ p_i$ is a minimal ideal which contains $I(Y) + (f)$. Otherwise, let $\mathfrak{p}$ be the minimal prime ideal satisfying $I(Y)+(f) \subseteq \mathfrak{p} \subsetneq \mathfrak{p}_i$. Then $V_i \subsetneq Z(\mathfrak{p})$, a irreducible closed subset of $X$. It contradicts to the fact that $V_i$ is irreducible component of $V \cap H$. Hence $ht(\mathfrak{p}_i) = n-r+1$ for all $1 \leq i \leq k$. It implies 
\[
\dim V_i = \dim A(V_i) = \dim k[x_1, \cdots, x_n] - ht(\mathfrak{p}_i)= r-1
\]
\qeds
\begin{exer}
	Let $\alpha \subseteq k[x_1, \cdots, x_n]$ be an ideal which can be generated by $r$ elements. Then every irreducible components of $Z(\alpha)$ has dimension $\geq n-r$.
\end{exer}
Let $Z(\alpha) = V_1 \cup \cdots \cup V_k$ be the decomposition of irreducible components, where $I(V_i) = \mathfrak{p}_i$ is prime ideal of $k[x_1, \cdots ,x_n]$. It implies $\alpha \subseteq \mathfrak{p}_1 \cap \cdots \cap \mathfrak{p}_k$. So $\alpha \subseteq \mathfrak{p}_i$ for each $i$. Hence $ht(\mathfrak{p}_i) \leq r$ by Krull principal ideal theorem. Therefore, the dimension of $V_i$ is greater than $n-r$.
\qeds
\section{Week2}
\begin{exer}
	Prove following statments
	\begin{itemize}
		\item If $T_1 \subseteq T_2$ are subsets of $S^h$, then $Z(T_1)\supseteq Z(T_2)$.
		\item If $Y_1 \subseteq Y_2$ are subsets of $\mathbb{P}^n$, then $I(Y_1) \supseteq I(Y_2)$.
		\item For any two subsets $Y_1,Y_2$ of $\mathbb{P}^n$, $I(Y_1 \cup Y_2) = I(Y_1) \cap I(Y_2)$.
		\item If $\mathfrak{a} \subseteq S$ is a homogeneous ideal with $Z(\mathfrak{a}) \neq \emptyset$, then $I(Z(\mathfrak{a}))= \sqrt{\mathfrak{a}}$.
		\item For any subset $Y \subset \mathbb{P}^n$, $Z(I(Y)) = \bar{Y}$.
	\end{itemize}

\end{exer}
Let $x \in Z(T_2)$, we have $f(x)=0$ for all $f \in Y_2$. Since all $g \in T_1$ are all in $Y_2$, we have $g(x)=0$. Hence $g \in Z(T_1)$. If $g \in I(Y_2)$, then $g$ vanishs on all $Y_2$, so on all $Y_1$. Hence $g \in I(Y_1)$. This also implies that both $I(Y_1)$ and $I(Y_2)$ contain $I(Y_1 \cup Y_2)$ since $Y_i \subseteq Y_1\cup Y_2$. Conversely, if $f \in I(Y_1) \cap I(Y_2)$, then $f$ vanishes on both $Y_1$ and $Y_2$, so $f \in I(V_1 \cup V_2)$ by definition.

If$f \in \sqrt{\mathfrak{a}}$, then there exists $n \geq 1$ such that $f^k \in \mathfrak{a}$. It implies every homogeneous part vanishes on $Z(\mathfrak{a})$. Let $f= f_1 + \cdots +f_n$ be the homogeneous decomposition of $f$. Then the homogeneous part of $f^k$ with degree $nk$ is $f^k_n$, so $f^k_n(P)=0$ for all $P \in Z(\mathfrak{a})$. Therefore, $f_n(P)=0$. By induction, we can conclude that $f_i(P)=0$ for all $i$. Hence $f \in I(Z(\mathfrak{a}))$. Conversely, if $f \in I(Z(\mathfrak{a}))$. By homogeneous Nullstellensatz, we have $f_i^{r_i} \in \mathfrak{a}$. Let $r= r_1+\cdots r_n$, then $f^r \in \mathfrak{a}$. Hence $I(Z(\mathfrak{a})) = \sqrt{\mathfrak{a}}$. 

Since $Y \subseteq Z(I(Y))$ and $Z(I(Y))$ is closed, we have $\bar{Y} \subseteq Z(I(Y))$. There is homogeneous ideal $\mathfrak{a}$ such that $\bar{Y} = Z(\mathfrak{a})$. From $Y \subseteq Z(\mathfrak{a})$, we have $I(Y) \subseteq I(Z(\mathfrak{a})) = \sqrt{\mathfrak{a}}$. Hence $\bar{Y}= Z(\mathfrak{a}) = Z(\sqrt{\mathfrak{a}}) \subseteq Z(I(Y))$. We now conclude that $Z(I(Y))= \bar{Y}$.
\qeds
\begin{exer}
	\begin{enumerate}[a)]
	\item There is a $1-1$ inclusion-reversing correspondence between algebraic sets in $\mathbb{P}^n$, and homogeneous radical ideals of $S$ not equal to $S_+$ does not occur in this correspondence, it is sometimes called the irrelevant maximal of $S$.
	\item An algebraic set $Y \subseteq \mathbb{P}^n$ is irreducible if and only if $I(Y)$ is a prime ideal.
	\item Show that $\mathbb{P}^n$ itself is irreducible.
	\end{enumerate}
\end{exer}
\begin{enumerate}[a)]
	\item If $\mathfrak{a}$ is radical homogeneous ideal of $S$ such that $Z(\mathfrak{a}) \neq 0$, then $I(Z(\mathfrak{a})) = \sqrt{\mathfrak{a}} =\mathfrak{a}$. If $Z(\mathfrak{a})=0$, then $I(Z(\mathfrak{a}))= I(\emptyset) = S$. $Z(\mathfrak{a}) =0$ implies $\mathfrak{a}= S$ or $S_+$. By assumption, $S_+$ is not in the correspondence, so $\mathfrak{a}= S$. Hence $I \circ Z$ is identity functor. Similarly, $Z \circ I$ is also identity. With previous exercise, this correspondence is inclusion-reversing.
	\item If $Y$ is irreducible, then for all $x,y \in I(Y)$, we can let $Y_1= Z(x) \cap Y$ and $Y_2 = Z(y) \cap Y$. Since $Y_1 \cup Y_2 = (Z(x) \cap Y) \cup (Z(y)\cap Y)= Z(xy) \cap Y = Y$, $Y_1= Y$ or $Y_2=Y$ by irreducible condition. It implies that $Z(x)=Y$ or $Z(y)=Y$. So $x \in I(Y)$ or $y \in I(Y)$. Conversely, suppose $I(Y)$ is prime. However, for any closed cover $Y_1 \cup Y_2 =Y$, we have $I(Y)= I(Y_1) \cap I(Y_2)$, therefore $I(Y_1)=I(Y)$ or $I(Y_2)=I(Y)$. Hence $Y_1 = Y$ or $Y_2 =Y$ since they are closed.
	\item $\mathbb{P}^n$ is algebraic set corresponding to radical homogeneous ideal $(0)$. It is prime ideal since $k[x_0, x_1 \cdots x_n]$ is integral domain. So $\mathbb{P}^n$ is irreducible from previous statement.
	\end{enumerate}
\qeds
\begin{exer}
	If $Y$ is a projective variety with homogeneous coordinate ring $S(Y)$, show that $\dim S(Y) = \dim Y +1$.
\end{exer}
$Y$ is projective variety, so let $Y= Z(\mathfrak{p}) \subseteq \mathbb{P}^n$ for some prime homogeneous ideal $\mathfrak{p}$. Hence any descending chain of closed subset of $Y$ corresponds to a descending chain of radical homogeneous ideal containing $\mathfrak{p}$ with the same length. However, $S_+$ is prime homogeneous ideal of $S$ which contains any non-zero ideal of $S$ but doesn't correspond to a algebraic set. Hence the $\dim S(Y) > \dim Y$. Since radical ideals contains $\mathfrak{p}$ except $S_+$ also correspond to closed subsets of $Y$, $\dim Y  \geq \dim S(Y)-1$. Hence $\dim S(Y) = \dim Y +1$.\qeds
\newpage
\section{Week 3}
\begin{exer}
	A regular function on projective variety is continuous map (view $k$ as affine line $\mathbb{A}^1$).
\end{exer}
Let $Y \subseteq P^n$ be a projective variety, then $Y$ can be covered by affine varieties $U_i=Y \cap A^n_i$, where $A^n_i$ are canonical affine coverings of $\mathbb{P}^n$. If $f\colon Y \to k$ be a regular function, then its restrictions $f_{U_i} \colon U_i \to k$ are continuous maps, and since $U_i$ are all open in $Y$, $f$ itself is continuous on $Y$. \qeds
\begin{exer}
	Let $\varphi: \mathbb{A}^1 \to C \hookrightarrow \mathbb{A}^2$ be curve defined as $ t \mapsto (t^2,t^3)$. Obviously, $\varphi$ is $1-1$ correspondence. Prove $\varphi$ is not isomorphism between varieties.
\end{exer}
Suppose the coordinate ring of $\mathbb{A}^2_k$ be $k[x,y]$. Then we have coordinate ring $A(C) \cong k[x,y]/(y^2-x^3)$. From definition of $\varphi$, we can write down its pull-back on coordinate rings
\[
\begin{aligned}
\varphi^* \colon A(C) &\to k[t]\\
f& \mapsto f \circ \varphi
\end{aligned}
\]
If $\varphi$ is isomorphism, then $\varphi^*$ is isomorphism between coordinate ring. Therefore it is also bijection between regular functions. Since $t$ is regular function on $\mathbb{A}^1_k$, it must have preimage $f$ such that $\varphi^*(f)=t$. This means that $f(t^2,t^3)=t$. $f$ is regular on $C$, so it is also regular at point $(0,0)$. Nearby $(0,0)$, $f$ can be written as
\[
\frac{\alpha(x,y)}{\beta(x,y)}
\]
where $\alpha ,\beta$ are polynomials in $k[x,y]$ and $\beta(0,0) \neq 0, \frac{\alpha(t^2,t^3)}{\beta(t^2,t^3)} = t$. It is impossible since $\alpha(t^2,t^3) = t \beta(t^2,t^3)$ cannot have term of degree 1.
Hence we can conclude $\varphi$ can not be an isomorphism otherwise it will induce isomorphism between coordinate ring. \\ \qeds
\begin{exer}
	Let $S= Z(y_0 y_2 - y_1^2)$ be the surface in $\mathbb{P}^2_k$ with the coordinates $(y_0:y_1:y_2)$. Let $\mathbb{P}^1_k$ be projective line with coordinate ring $k[x_0,x_1]$. Consider morphism \[
	\begin{aligned}
	\varphi: \mathbb{P}^1_k &\to S \subset \mathbb{P}^2_k\\
	(x_0:x_1) &\mapsto (x_0^2: x_0x_1: x_1^2)
	\end{aligned}\] and show that $\varphi$ is isomorphism.
\end{exer}
It is well-defined regular morphism since $(x_0^2)(x_1^2) - (x_0 x_1)^2=0$ and with polynomial in each component. We can see that $\varphi$ is bijection and $\varphi^{-1}$ is defined as
\[
\varphi^{-1} \colon (y_0:y_1:y_2) =\begin{cases}
(y_0: y_1)& \text{if } y_0 \neq 0\\
(y_1: y_2)& \text{if } y_2 \neq 0\\
\end{cases} 
\]
It is well-defined since we have $(y_0: y_1) = (\frac{y_0}{y_1}:1) = (\frac{y_1}{y_2}:1) = (y_1:y_2)$ when neither $y_0$ or $y_2$ are  equal to $0$. Moreover, it is regular map because it is regular on open subsets $S \cap \{y_0 \neq 0\}$ and $S \cap \{y_2 \neq 0\}$. With the fact that $\varphi \circ \varphi^{-1} = id_{S}, \varphi^{-1} \circ \varphi = id_{\mathbb{P}^1_k}$, we can conclude that $\varphi$ is isomorphism. 

\qeds
\section{Week 4}
\begin{exer}
	Let $Y$ be an affine variety in $\mathbb{A}^n_k$. Show that $K(Y) \cong K(\mathcal{O}_{Y,p})$ for all $p \in Y$.
\end{exer}
Since $Y$ is affine variety, we have $\mathcal{O}_{Y,p} \cong A(Y)_{m_p}$ for all point $p \in Y$. Therefore, $K(\mathcal{O}_{Y,p})$ is fraction field of local ring $A(Y)_{m_p}$. Moreover, $K(Y)$ is fraction field of coordinate ring $A(Y)$. Hence we have following commutative localization diagram
\[
\begin{tikzcd}
A(Y) \ar[r,"l_{m_p}"] \ar[d,"l"] & A(Y)_{m_p} \ar[d,"l'"]\\
K(A(Y)) & K(A(Y)_{m_p})\\
\end{tikzcd}
\]
$l_{m_p}$ maps non-zero divisor of $A(Y)$ to non-zero divisor, so $l' \circ l_{m_p}$ maps non-zero divisors to units. Therefore, with universal property of localization, there is unique homomorphism $g \colon K(A(Y)) \to K(A(Y)_{m_p})$ making this diagram commute. Also, with universal property of $l_{m_p}$, there is unique morphism $h$ making following diagram commute
\[
\begin{tikzcd}
&A(Y) \ar[r,"l_{m_p}"] \ar[d,"l"]& A(Y)_{m_p} \ar[ld,"h"] \\
&K(A(Y))& \\
\end{tikzcd}
\]
Now we get a new localization diagram
\[
\begin{tikzcd}
& &A(Y)_{m_p} \ar[d,"l'"] \ar[dl,"h"]\\
&K(A(Y)) & K(A(Y)_{m_p}) \ar[l,dashed,"\exists !f"]\\
\end{tikzcd}
\]
Hence we can conclude that $K(A(Y)) \cong K(A(Y)_{m_p})$, which implies that $K(Y) \cong K(\mathcal{O}_{Y,p})$ for all $p \in Y$. \qeds
\begin{exer}
	Prove that for any integer $0 \leq i \leq n$
	\[
	K(\mathbb{P}^n_k) \cong k(x_0/x_i, \cdots, \widehat{x_i/x_i}, \cdots, x_n/x_i)
	\]
\end{exer}
For any variety $Y$, if $U$ is open subvariety of $Y$, then by definition $K(Y) \cong K(U)$. More precisely, we can send $\langle f, V \rangle$ to $\langle f, V \cap U \rangle$ and it is an well-defined isomorphism between function fields. Therefore, for any $0 \leq i \leq n$, we have $K(A_i) \cong K(\mathbb{P}^n_k)$, where $A_i$ is affine cover which is isomorphic to $\mathbb{A}^n_k$ with coordinate ring $k[x_0/x_i, \cdots, \widehat{x_i/x_i}, \cdots, x_n/x_i]$. With the conclusion in last exercise, we can conclude that 
\[
K(\mathbb{P}^n_k) \cong K(A_i) \cong k(x_0/x_i, \cdots, \widehat{x_i/x_i},\cdots, x_n/x_i)
\]
\begin{exer}
	Equation $x_0^2 + x_1^2 + x_2^2=0$ defines a conic $X \hookrightarrow \mathbb{P}_k^2$. Find $t \in K(X)$ such that $K(X) \cong k(t)$ is a transcendental extension of $k$ with degree 1.
\end{exer}
$K(X)$ is equal to function field of affine open subset defined by
\begin{equation}
\label{eq1}
	(\frac{x_1}{x_0})^2 + (\frac{x_2}{x_0})^2 = -1
\end{equation}
Consider linear transform
\begin{align}
y_1 = \frac{x_1}{x_0} + i \frac{x_2}{x_0}& &
y_2 = \frac{x_1}{x_0} - i \frac{x_2}{x_0}
\end{align}
Then the equation \ref{eq1} becomes
\begin{equation}
	y_1 y_2 =-1
\end{equation}
Hence $K(X) = k(y_1,y_2) = k(y_1)$ since $y_2 =-1/y_1$. Therefore $y_1$ is the required $t$. \qeds
\newpage
\section{Week5}
\begin{exer}
	Suppose $S = Z(xy-zw) \subset \mathbb{P}^3_k = \proj k[x,y,z,w]$. Let $H \subset \mathbb{P}_k^3$ the hyperplane in $\mathbb{P}_k^3$ defined by $x=0$. It is isomorphic to $\mathbb{P}_k^2$.
	\[\begin{aligned}
		\varphi \colon S &\dashrightarrow H \\
		(x: y: z: w) &\mapsto (0: y: z-x: w)\\
	\end{aligned}\]
	Prove that $\varphi$ is birational map.
\end{exer}
\begin{proof}
	Actually, we define its rational inverse as 
\[
\begin{aligned}
\varphi^{-1} \colon H &\dashrightarrow S\\
(0: s_1: s_2: s_3) & \mapsto (\frac{s_2 s_3}{s_1 -s_3}: s_1 : \frac{s_1 s_2}{s_1 - s_3}: s_3)
\end{aligned}
\]
It is defined on $H$ on the open subset $\{s_1 \neq s_3\} \cap H$ and $\varphi^{-1}(H) \cap (S \cap A_i) \neq \emptyset$ for all canonical affine covers $A_i$ of $\mathbb{P}_k^3$. It means $\varphi^{-1}$ is well-defined dominant rational map on $H$. Furthermore, its compositions with $\varphi$ are all identities as rational maps since
\[
 \varphi((\frac{s_2 s_3}{s_1 - s_3}: s_1 : \frac{s_1 s_2}{s_1 -s_3}:s_3)) = (0: s_1: \frac{s_2(s_1-s_3)}{s_1 - s_3}: s_3) = (0: s_1: s_2 : s_3)
\]
and
\[
\begin{aligned}
\varphi^{-1}((0:y:z-x:w)) &= (\frac{(z-x)w}{y-w}: y: \frac{y(z-x)}{y-w}: w)\\
 &= (\frac{xy-xw}{y-w}: y: \frac{yz- zw}{y-w}: w)\\
 & =(x:y:z:w)
\end{aligned}
\]
\end{proof}
\begin{exer}
	If $Y, Z \subset \mathbb{A}^2_k$ are two distinct curves given by equations $f=0,  g=0$, and if $p \in Y\cap Z$, we define the intersection multiplicity $(Y\cdot Z)_p$ at point $p$ to be the length of the $\mathcal{O}_p$-module $\mathcal{O}_p/(f,g)$. Show that 
	\begin{itemize}
		\item $(Y\cdot Z)_p$ is finite and $(Y\cdot Z)_p \geq \mu_p(Y) \mu_p(Z)$
		\item  If $p \in Y$ show that for almost all lines $l$ through $p$, we have $(l \cdot Y)_p = \mu_p(Y)$.
		\item If $Y$ is a curve of degree $d$ in $ \mathbb{P}_k^2$, and if $l$ is a line in $\mathbb{P}_k^2$, $l \neq Y$, show that $(l \cdot Y) =d $. Here we define $(l \cdot Y) = \sum_p (l \cdot Y)_p$ taken over all points $p \in l \cap Y$, where  $(l\cdot Y)_p$ is defined using a suitable affine cover of $\mathbb{P}_k^2$.
	\end{itemize}
\end{exer}
\begin{proof}
	\begin{itemize}
	\item To show $(Y \cdot Z)_p$ is finite, we just need to prove $\mathcal{O}_p$-module $\mathcal{O}_p/(f,g)$ is of finite length. This is equivalent to prove that $\mathcal{O}_p/(f,g)$ is both Artinian and Noetherian module. Since $\mathcal{O}$ is localization of Noether ring $k[x,y]$, it is also Noether. The natural morphism $\mathcal{O}_p \twoheadrightarrow \mathcal{O}_p/(f,g)$ implies that $\mathcal{O}_p/(f,g)$ is finitely generated $\mathcal{O}_p$-module, so it is Noetherian module. Furthermore, $\mathcal{O}_p/(f,g)$ is of Krull dimension $0$, because of  $\dim \mathcal{O}_p=2$ together with Krull's principal ideal theorem. Therefore $\mathcal{O}_p/(f,g)$ is Artinian ring  since $\mathcal{O}_p/(f,g)$ is Noether ring and it is also Artinian module as $\mathcal{O}_p$-module. Suppose $Y$ and $Z$ intersect at $(0,0)$.
	\item Make a linear change of the coordinates so that $p=(0,0)$. Then $\mathcal{O}_p  = k[x,y]_{(x,y)} \cong k[x,y]$. Therefore, any line through $p$ can be defined by equation $ax+by =0$. If $b \neq 0$, then $y= -\frac{a}{b} x$. In this case, $\mathcal{O}_p/(f,ax+by) \cong k[x]/(f(x, -a/b x))$. Let $\mu_p(Y) = r$ and $f$ is of degree $d$, then $f$ can be written in the form
	\[
	f(x, -\frac{a}{b} x) = c_0 x^r + c_1 x^{r+1} + \cdots + c_{d-r}x^{d}= c_0 x^r(1+ \cdots + c_{d-r} x^{d-r})
	\]
	$1+ \cdots c_{d-r} x^{d-r}$ can be viewed as polynomial of $\frac{a}{b}$ with coefficients in $k[x]$, so there are only finite $a$ such that $1+ \cdots c_{r-d}x^{r-d} = 0$. Hence $\mathcal{O}_p/(f,ax+by) \cong k[x]/(c_0 x^r)$ since $c_0 x^r$ and $1+ \cdots c_{d-r}x^{d-r}$ are coprime except finite number of $a$. Therefore, length of $\mathcal{O}_p/(f,ax+by)=r$. If $b=0$, it determines only line in $\A^2_k$. Hence we reach the conclusion.
	\item $Y$ is curve in $\p_k^2$, so $Y$ is determined by homogeneous polynomial $f$. Let $A_i =\{x_i\neq 0\} \cap \p^2$ be the three affine covers of $\p_k^2$. Under linear transformation, we can make $l$ through $(0:0:1)$, hence equation of $l$ can be written as
	\[
	ax+by =0
	\]
	In this case $l$ can be covered by $A_1 \cup A_3$ if $b \neq 0$. Then
	\begin{equation}
	\label{eq1}
(l \cdot Y) =\sum_{p \in A_1 \cap l \cap Y} (l\cdot Y)_p + (l\cdot Y)_{(0:0:1)}
	\end{equation}
	
	Suppose $Y$ is determined by homogeneous polynomial $f$. On $A_1$, the $Y\cap l$ is determined by 
	\[
	f(1, -\frac{b}{a}, z/x) =0
	\]
	Let $F(t)= f(1, -\frac{1}{b},t)$ is of degree $r$. Then the first term of right part of $\ref{eq1}$ is equal to $r$. With last exercise, we know $(l\cdot Y)_{(0:0:1)} = \mu_{(0:0:1)}(Y)=d-r$.
	Hence, we can conclude that $(l\cdot Y) = d$
	\end{itemize}
\end{proof}

\section{Week6}
\begin{exer}
	Let $Y$ be the curve $y^2= x^3-x$ in $\mathbb{A}_k^2$, and assume that the characteristic of the base field $k$ is not $2$. We can find that $Y$ is nonsingular and $A= A(Y)$ is integral closed domain. Furthermore, $k[x] \subset K=K(Y)$ is a polynomial subring and $A$ is the integral closure of $k[x]$ in $K$. There is an automorphism $\sigma:A \to A$ which sends $y$ to $-y$ and leaves $x$ fixed. We define the norm $N$ of $A$ which maps $a$ to $N(a) = a \cdot \sigma(a)$ for any $a$ in $A$. 
	
	Using this norm, show that the units in $A$ are precisely the non-zero elements of $k$. Show that $x$ and $y$ are irreducible elements of $A$. Show that $A$ is not a UFD.
\end{exer}

\begin{exer}
	We know that if $X$ is a quasi-projective curve and $\varphi \colon X\backslash P \to Y$ where $P\in X$ and $Y$ is a projective variety, there is a unique morphism $\bar{\varphi} \colon X \to Y$ extending $\varphi$. But it is not true when $\dim X \geq2$.
	
	Let $\varphi \colon \mathbb{P}_k^2 \dashrightarrow \mathbb{P}_k^2$ be the Cremona transformation which is defined on $\mathbb{P}_k^2 -\{(1:0:0),(0:1:0),(0:0:1)\}$ and sends $(x:y:z)$ to $(yz:xz:xy)$, show that $\varphi$ can not extend to a morphism from $\p_k^2$ to $\mathbb{P}_k^2$.
\end{exer}
\begin{proof}
	If $\varphi$ can be extended to a morphism $\tilde{\varphi}$ on $\p_k^2$. Consider the embedding of projective line $i_1 \colon \p_k^1 \hookrightarrow \p_k^2$ by sending $(x:y)$ to $(x:y:0)$. Then points $(1:0:0)$ and $(0:1:0)$ are in $i_1(\p_k^1)$. If $p \in \p_k^1$ is not equal to $(1:0)$ or $(0:1)$, then $\tilde{\varphi}\circ i_1(p) = (0:0:xy)=(0:0:1)$. Hence $\p_k^1 -(1:0)-(0:1) \subseteq S= (\tilde{\varphi}\circ i_1)^{-1}((0:0:1))$. It implies
	\[
	\p_k^1= \overline{\p_k^1-(1:0)-(0:1)} \subseteq S \subseteq \p_k^1
	\]
	Therefore, $S= \p_k^1$. This means $\tilde{\varphi}((1:0:0))= \tilde{\varphi}((0:1:0)) = (0:0:1)$. However, if we take another embedding $i_2$ of projective line into $\p_k^2$,sending $(x:y)$ to $(x:0:y)$, then we can get $\tilde{\varphi}((1:0:0))=\tilde{\varphi}((0:0:1)) = (0:1:0)$. So we cannot define the value of $(1:0:0)$ properly. Hence we can conclude that $\varphi$ cannot extend to a morphism on $\p_k^2$.
\end{proof}
\begin{exer}
	Think $\p_k^1$ as $\A_k^1 \cup \infty$. Then we define a fractional linear transformation of $\p_k^1$ by sending $x \mapsto (ax+b)/(cx+d)$, for $a,b,c,d \in k$ and $ad-bc\neq0$
	\begin{itemize}
		\item Show that a fractional linear transformation induces an automorphism of $\p^1_k$ (i.e., an isomorphism of $\p_k^1$ with itself). We denote the group of all these fractional linear transformations by $\pgl(1)$.
		\item Let $\aut(\p_k^1)$ denote the group of all automorphisms of $\p_k^1$. Show that $\aut(k(x))$, the group pf $k$-automorphisms of the fractional field $k(x)$.
		\item Now show that every automorphism of $k(x)$ is a fractional linear transformation, and deduce that $\pgl(1) \to \aut(\p_k^1)$ is an isomorphism.
	\end{itemize}
\end{exer}
\begin{proof}
	\begin{itemize}
		\item Let $\infty=(1:0) \in \p_k^1$. Then fractional linear transformation $x \mapsto (ax+b)/(bx+d)$ can be written in coordinates as\[
		(x:y) \mapsto (ax+by:cx+dy)
		\]
		It is well-defined regular morphism since its each component is polynomial of $x,y$ and $(ax+by:cx+dy) \neq (0:0)$ since $ad-bc \neq 0$. Take the inverse matrix of
		\[
		\begin{pmatrix}
		a&b\\
		c&d\\
		\end{pmatrix}
		\]
		We get the inverse of the fractional linear transformation
		\[
		(m:n) \mapsto (\frac{dm-bn}{ad-bc}: \frac{-cm+an}{ad-bc})
		\]
		Hence it is automorphism on $\p_k^1$.
		\item First, we note that function field $K(\p_k^1) \cong K(\A_k^1) \cong k(x)$. Taking function field of variety is functional, hence automorphism of $\p_k^1$ induces automorphism of $k(x)$ and it is $k$-linear since it actually induces morphism of $k$-algebras. Conversely, any $k$-automorphism $\varphi$ of $k(x)$ is determined by the image $\varphi(x)$. Let $f= \varphi(x) \in k(x)$, then we define $\phi \colon \p_k^1 \to \p_k^1$ as follows 
		\[
		\phi((x:y)) = \begin{cases}
		(f(x/y):1)& \text{ if } y \neq 0\\
		(1:f(y/x))& \text{ if } x \neq 0\\
		\end{cases}
		\]
		Or equivalently $\phi(x:y) = (f(x): f(y))$. Since $f \in k(x)$ and $\varphi$ has inverse, $\phi$ is well-defined automorphism of $\p_k^1$. Furthermore, we have $K(\phi) = \varphi$, which is induced isomorphism by $\p_k^1$-automorphism on its function field. Hence we can conclude that $\aut(\p_k^1) \cong \aut(k(x))$.
		\item Since $f \in k(x)$, it can be written as
		\[
		f(x)= \frac{a_0 + a_1 x + \cdots + a_m x^m}{b_0 + b_1 x+ \cdots + b_n x^n}
		\]
		Let $y= f(x)$. Because $\varphi$ is isomorphism, $f(x)$ must have form
		\[
		f(x) = \frac{a_0 + a_1 x}{b_0 +b_1 x}
		\]
		Since $\varphi$ is determined by $f(x)$, there is one-to-one correspondence between $\pgl(1)$ and $\aut(k(x))$ and it is isomorphism of groups.
	\end{itemize}
\end{proof}
\newpage
\section{Week 8}
\begin{exer}
	\begin{itemize}
		\item Find the degree of the $d$-uple embedding of $\mathbb{P}^n$ in $\mathbb{P}^N$ where $N = \binom{n+d}{n}-1$.
		\item Find the degree of the Serge embedding of $\mathbb{P}^r \times \mathbb{P}^s$ in $\mathbb{P}^N$ where $N= r+s +rs$.
	\end{itemize}
\end{exer}
\begin{proof}
	The $d$-uple embedding is defined as
	\[
	\rho_d \colon P=(a_0 : \cdots: a_n) \mapsto (M_0(a): \cdots : M_N(a))
	\]
	where $M_i$ are all monomial of degree $d$ with variables $a_0, \cdots ,a_n$. Let $S(\mathbb{P}^N) = k[y_0, \cdots, y_N]$ and $S(\mathbb{P}^n)=k[x_0,\cdots,x_n]$. $\rho_d$ induce $k$-algebra homomorphism
	\[
	\varrho_d \colon y_i \mapsto M_i(x_0, \cdots, x_n)=x_0^{i_0} \cdots x_n^{i_n}
	\]
	Suppose kernel of this morphism is $\mathfrak{a}$. It is homogeneous prime ideal corresponds to image of $\rho_d$. Hence the coordinate ring of image $M$ of $\rho_d$ is $S(M)= k[y_0, \cdots, y_N]/\mathfrak{a} \cong \im \varrho_d$. It means that 
	\[
	S(M) = \bigoplus_{l=0}S(M)_l\cong \bigoplus_{l=0} k[x_0, \cdots, x_n]_{ld}
	\]
	where $S(M)_l \cong k[x_0, \cdots, x_n]_{ld}$ is $\binom{ld+n}{n}$ dimensional vector space. Hence the degree $M$ is $d^n$.
	
	Similarly as $d$-uple embedding, the ideal of image of Segre embedding is kernel of 
	\[
	 \begin{aligned}
	 \theta \colon k[z_{00}, \cdots, z_{ij}, \cdots, z_{rs}] &\to k[x_0, \cdots, x_r; y_0, \cdots, y_s]\\
	 &z_{ij} \mapsto x_i y_j
	 \end{aligned}
	\]
	We denote its kernel by $\mathfrak{b}$, hence $S(X)\cong k[z_{00}, \cdots, z_{ij}, \cdots, z_{rs}]/\mathfrak{b} \cong \im \theta$. Hence \[
	S(X) = \bigoplus_{j=0} S(X)_j \cong \bigoplus_{j=0} k[x_0y_0, \cdots, x_r y_s]_{j} \cong \bigoplus_{j=0} k[x_0, \cdots, x_r]_j \times k[y_0, \cdots, y_s]_j
	\]
	where $x_iy_j$ is of grade $1$. Hence $\dim_k S(X)_j = \binom{j+r}{j} \binom{j+s}{j}$. 
\end{proof}
\begin{exer}
	Show that an algebraic set $Y$ of pure dimension $r$ (i.e., every irreducible component of $Y$ has dimension $r$) has degree $1$ if and only if $Y$ is a linear variety.
\end{exer}
\begin{proof}
	If $Y$ is linear variety of codimenision $1$ in $\mathbb{P}^n$, then $Y$ is variety in $\mathbb{P}^n$ determined by single linear equation, so it is just hyperplane in $\mathbb{P}^n$. Then we assume that all linear varieties of codimension $k$ is of degree $1$. For some linear variety $Y^{k+1}$ of codimension $k+1$, it intersection of a hyperplane $H$ and a linear variety $Y^k$ of codimension $k$, so with theorem 7.7, we have 
	\[
	i(Y^k,H;Y^{k+1})\deg Y^{k+1} = \deg Y^k \deg H =1
	\]
	this means that $i(Y^k,H;Y^{k+1}) = \deg Y^{k+1
	} =1$. Hence a linear variety is of degree $1$.
Conversely, let $Y^1= Z(I)$ be some algebraic set in $\mathbb{P}^n$ of pure codimension $1$ and $n \geq 2$. Assume $Y = \cup_i Y_i$ it the decomposition of irreducible components of $Y$. For each $Y_i$, we can choose two distinct points $P,Q$ and a hyperplane $H$ in $\mathbb{P}^n$ which contains this two points. The theorem 7.7 implies that the degree of $Y_i$ is greater than $2$ if $Y_i$ is not contained in $H$. So if $Y_i$ is of degree $1$, then $Y_i$ is contained in a hyperplane in $\mathbb{P}^n$. Hence $Y_i$ is just this hyperplane since they are irreducible and of same dimension $n-1$. So each components of $Y^1$ is hyperplane in $\mathbb{P}^n$, hence $Y^1$ itself is irreducible hyperplane since every two hyperplane intersects with each other. If $n=1$, then $Y$ is just single point of degree $1$ or $\mathbb{P}^1$ itself, of course it is linear. Now, by induction, we can assume it is true for algebraic set of pure codimension $k$ with degree $1$ that it is linear. For a algebraic set $Y^{k+1}$ in projective space $\mathbb{P}^n$. Suppose $Y^{k+1}_i$ is one of its irreducible components. We can also choose two distinct points $P,Q$ in $Y^{k+1}_i$ and it also implies $Y^{k+1}_i$ is contained in a hyperplane $H$. Take an isomorphism $H \cong \mathbb{P}^{n-1}$, then $Y^{k+1}_i$ can be viewed as variety in $\mathbb{P}^{n-1}$ with codimension $k$ of degree $1$ ( since all mentioned embeddings are closed embeddings). Hence it is linear variety. We have already known that degree of algebraic set is sum the degrees of its irreducible components. Hence $Y^{k+1}$ is irreducible, and furthermore it is linear variety.
\end{proof}
\begin{exer}
	Let $Y^r$ be a $r$-dimensional variety of degree $2$. Show that $Y$ is contained in a linear subspace $L$ of dimension $r+1$ in $\mathbb{P}^n$. Thus $Y$ is isomorphic to a quadratic hypersurface in $\mathbb{P}^{r+1}$.
\end{exer}
\begin{proof}
	Choose non-singular point $x \in Y^r$. Then the union of all lines  through $x$ and some other points in $Y^r$ is an $r+1$-dimensional variety of degree $1$, denoted by $P$. Therefore, combine statement of previous exercise, we can conclude that $P$ is linear.
	
	Or we can choose three distinct points in $Y^r$, then we can find a hyperplane $H$ in $\mathbb{P}^n$ that contains such three points. Since $Y^r$ is of degree $2$, then $Y^r$ musted be contained in $H$. With linear transformation $H \cong \mathbb{P}^{n-1} \hookrightarrow \mathbb{P}^n$. Hence $Y^r$ is variety in $\mathbb{P}^{n-1}$ with degree $2$. Repeating this process, we can closed embed $Y^r$ in $\mathbb{P}^{r+1} \hookrightarrow \mathbb{P}^n$ with linear transformation. Therefore $Y^r$ is contained in a $r+1$-dimensional linear subspace of $\mathbb{P}^n$.
	Hence $P$ is linear subspace of $\mathbb{P}^n$ of dimension $r+1$ which contains $Y^r$. This implies that $Y^r$ is hypersurface of some $\mathbb{P}^{r+1} \hookrightarrow \mathbb{P}^n$ with degree $2$. It is defined by polynomial of degree $2$. Therefore we can conclude that it is quadratic in $\mathbb{P}^{r+1}$. 
\end{proof}
\section{Week 9}
\begin{exer}
	Let $P=(0,0)$, $C = (x^2+y^2)^2 + 3x^2y -y^3$ and $D = (x^2+y^2)^3 -4x^2y^2$, directly calculate the intersection number $i(P, C\cap D)$.
\end{exer}
\begin{proof}
	\begin{itemize}
		\item We need to compute ideal $((x^2+y^2)^2+3x^2y-y^3, (x^2+y^2)
^3 -4x^2y^2)$ in local ring $k[x,y]_{(x,y)} \cong k[x,y]$. See the attached draft.
\item Let $f= (x^2+y^2)^2 + 3x^2y -y^3$ and $g=(x^2+y^2)^3 -4x^2y^2$. Homogenizing them, we get\[
\begin{aligned}
&F= x^4 +2x^2y^2+y^4 + 2x^2yz -y^3z\\
&G= x^6+3x^4y^2 +3x^2y^4 + y^6 -4x^2y^2z^2\\
\end{aligned}
\]
After linear transformation $(x:y:z) \mapsto (z:x:y)$, the equation of $C$ and $D$ in $\mathbb{P}^2$ becomes
\[
\begin{aligned}
&\tilde{F}= y^4 +2y^2z^2+z^4 + 2y^2zx -y^3x\\
&\tilde{G}= y^6+3y^4z^2 +3y^2z^4 + z^6 -4y^2z^2x^2\\
\end{aligned}
\]
	\end{itemize}
Consider them as polynomial in $k[x,y][z]$. Compute their resultant with Mathematica
\[
\text{Res}(\tilde{F},\tilde{G},z)= 256 x^{10} y^{14} - 768 x^9 y^{15} + 224 x^{8} y^{16} + 432 x^7 y^{17} + 125 x^6 y^{18}
\]
\end{proof}
Hence $i(p,C \cap D)$ is degree of $y$ in this resultant, it is equal to $14$.
\begin{exer}
	Suppose $\Char k \neq 2$ and let $X$ be a nonsingular projective curve of genus1, then for any effective divisor $D$ on $X$ we have $l(D) = \deg D$ and thus for $P \in X$ a point we have $l(P)=1$. There is no element in $K(X)$ has one pole exactly at $P$. Hence, $L(2P)=(1,x) \subseteq K(X)$, the bracket here is linear span with given base, where $P$ is two order point of $x \in K(X)$ and $L(3P) = (1,x,y) \subseteq K(X)$ where $P$ is three order pole of $y$.
	\begin{itemize}
		\item Show $L(4P)= (1,x,y,x^2)$ and $L(5P)= (1,x,y,x^2,xy)$.
		\begin{proof}
			For two degree terms $x^2, xy, y^2$, we have\[
			\begin{aligned}
			\ord_p(x^2)= 4& &\ord_p(xy) = 2+ 3 =5 & &\ord_p(y^2)= 2\times 3=6
			\end{aligned}
			\]
			Hence $x^2 \in L(4P)$ and others are not.
			It also implies $L(5P) = (1,x,y,x^2,xy)$.
		\end{proof}
	\item Derive the Weierstrass equation for genus $1$ curves.
	\begin{proof}
		For three degree terms $x^3,x^2y,xy^2,y^3$, we have \[
		\begin{aligned}
		\ord_P(x^3) =6 & & \ord_P(x^2y)= 7 & & \ord_P(xy^2)=8 & &\ord_P(y^3)=9
		\end{aligned}
		\] hence we have $(1,x,y,x^2,y^2,xy, x^3) \subseteq L(6P)$. While Riemann-Roch tells us $L(6P) \leq 6$, these generators are linear dependent. Therefore we have
		\[
		y^2+b_0xy + b_1 y = a_3 x^3 + a_2 x^2 + a_1 x + a_0
		\]
		with suitable linear transformation we can assume
		\[
		y^2= x^3 +a x +b
		\]
		This is Weierstrass equation for genus $1$ curve.
		Since we have assume that $X$ is non-singular, 
		\[
		x^3+ax+b =0
		\]
		should has three distinct roots. Otherwise, if $x^3+ax+b = (x-\alpha)^2(x-\beta)$, then the Jacobian of $y^2-x^2-ax-b$ at point $(\alpha,0)$ vanishes. Therefore let
		\[
		y^2 = \prod_{i=1}^3(x-\tau_i)
		\]
		hence with embedding $(x,y) \mapsto (1:x:y)$, $X$ is birational to a cubic curve in $\mathbb{P}^2$ defined by
		\[
		zy^2= x^3+axz^2+bz^3
		\]
	\end{proof}
	\end{itemize}
\end{exer}
\end{document}
