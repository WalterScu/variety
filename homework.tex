% !TEX program=luatex
\documentclass[12pt,a4paper]{article}
\usepackage{luatexja-fontspec}
\setmainjfont{FandolSong}
\usepackage{amsmath}
\usepackage{amsthm}
\usepackage{amssymb}
\usepackage{amsfonts}
\usepackage{enumerate}
\usepackage[colorlinks]{hyperref}
\usepackage{tikz-cd}
\usepackage{geometry}
\geometry{left=2cm,right=1cm, top=3cm,bottom=2cm}
\theoremstyle{definition}
\newtheorem{secdefn}{Definition}[subsection]
\newtheorem{exer}{Exercies}[subsection]
\newcommand*{\qeds}{\hfill\ensuremath{\clubsuit}}
\DeclareMathOperator{\spec}{Spec}
\begin{document}
\noindent
{\LARGE\underline{\textbf{Algebraic Geometry}}}\\
{\hfill\large  \underline{\textbf{邹海涛}} \\
	\hfill ID: 17210180015}\\
\section{Week 1}
\begin{exer}
	Any nonempty open subset of an irreducible topological space is dense and irreducible.
\end{exer}
Let $X$ be an irreducible space and $U \hookrightarrow X$ be an nonempty open subset of $X$. Let $V_1= X\backslash U $ and $V_2= \bar{U}$. Then we have
\[
V_1 \cup V_2 \supseteq (X \backslash U) \cup U =X
\]
Since $X$ is irreducible and $V_1, V_2$ are closed subsets, we have $V_1 = \emptyset$ or $V_2 = X$. That means $\bar{U} =X$. Hence $U$ is dense open subset of $X$.
We can further prove that $X$ is irreducible if any nonempty open subset of $X$ is dense. Otherwise, $X$ is reducible, then $X = V_1 \cup V_2$ where $V_1$ and $V_2$ are non-trivial closed subset of $X$. Then $(X\backslash V_1) \cap (X\backslash V_2) = \emptyset$. It implies $X\backslash V_1$ is non-empty open subset of $X$ which is not dense. Hence we can conclude that any nonempty open subset of irreducible space $X$ is irreducible because its open subsets are all dense in $X$, also in itself.
\qeds
\begin{exer}
	Let $Y$ be an affine variety of dimension $r$ in $\mathbb{A}^n$. Let $H$ be a hypersurface in $\mathbb{A}^n$ and assume that $Y \subsetneq H$. Then every irreducible component of $Y \cap H$ has dimension $r-1$.
\end{exer}
Suppose $H$ be irreducible. It means ideal of $H$ is prime ideal $(f)$ of $k[x_1, \cdots, x_n]$. Let $Y \cap H = V_1 \cup \cdots \cup V_k$ be the irreducible components decomposition and ideal of $V_i$ is $\mathfrak{p}_i$. Hence we have
\[
I(V \cap H) = \mathfrak{p}_1 \cap \cdots \cap \mathfrak{p}_k
\]
Since $Y \varsubsetneq H$, we have $f \notin I(Y)$. Hence the minimal prime ideals of $I(Y) + (f)$ is with height $n-k+1$ by Krull principal ideal theorem, since $I(Y)$ is of height $n-r$. We claim that $ p_i$ is a minimal ideal which contains $I(Y) + (f)$. Otherwise, let $\mathfrak{p}$ be the minimal prime ideal satisfying $I(Y)+(f) \subseteq \mathfrak{p} \subsetneq \mathfrak{p}_i$. Then $V_i \subsetneq Z(\mathfrak{p})$, a irreducible closed subset of $X$. It contradicts to the fact that $V_i$ is irreducible component of $V \cap H$. Hence $ht(\mathfrak{p}_i) = n-r+1$ for all $1 \leq i \leq k$. It implies 
\[
\dim V_i = \dim A(V_i) = \dim k[x_1, \cdots, x_n] - ht(\mathfrak{p}_i)= r-1
\]
\qeds
\begin{exer}
	Let $\alpha \subseteq k[x_1, \cdots, x_n]$ be an ideal which can be generated by $r$ elements. Then every irreducible components of $Z(\alpha)$ has dimension $\geq n-r$.
\end{exer}
Let $Z(\alpha) = V_1 \cup \cdots \cup V_k$ be the decomposition of irreducible components, where $I(V_i) = \mathfrak{p}_i$ is prime ideal of $k[x_1, \cdots ,x_n]$. It implies $\alpha \subseteq \mathfrak{p}_1 \cap \cdots \cap \mathfrak{p}_k$. So $\alpha \subseteq \mathfrak{p}_i$ for each $i$. Hence $ht(\mathfrak{p}_i) \leq r$ by Krull principal ideal theorem. Therefore, the dimension of $V_i$ is greater than $n-r$.
\qeds
\section{Week2}
\begin{exer}
	Prove following statments
	\begin{itemize}
		\item If $T_1 \subseteq T_2$ are subsets of $S^h$, then $Z(T_1)\supseteq Z(T_2)$.
		\item If $Y_1 \subseteq Y_2$ are subsets of $\mathbb{P}^n$, then $I(Y_1) \supseteq I(Y_2)$.
		\item For any two subsets $Y_1,Y_2$ of $\mathbb{P}^n$, $I(Y_1 \cup Y_2) = I(Y_1) \cap I(Y_2)$.
		\item If $\mathfrak{a} \subseteq S$ is a homogeneous ideal with $Z(\mathfrak{a}) \neq \emptyset$, then $I(Z(\mathfrak{a}))= \sqrt{\mathfrak{a}}$.
		\item For any subset $Y \subset \mathbb{P}^n$, $Z(I(Y)) = \bar{Y}$.
	\end{itemize}

\end{exer}
Let $x \in Z(T_2)$, we have $f(x)=0$ for all $f \in Y_2$. Since all $g \in T_1$ are all in $Y_2$, we have $g(x)=0$. Hence $g \in Z(T_1)$. If $g \in I(Y_2)$, then $g$ vanishs on all $Y_2$, so on all $Y_1$. Hence $g \in I(Y_1)$. This also implies that both $I(Y_1)$ and $I(Y_2)$ contain $I(Y_1 \cup Y_2)$ since $Y_i \subseteq Y_1\cup Y_2$. Conversely, if $f \in I(Y_1) \cap I(Y_2)$, then $f$ vanishes on both $Y_1$ and $Y_2$, so $f \in I(V_1 \cup V_2)$ by definition.

If$f \in \sqrt{\mathfrak{a}}$, then there exists $n \geq 1$ such that $f^k \in \mathfrak{a}$. It implies every homogeneous part vanishes on $Z(\mathfrak{a})$. Let $f= f_1 + \cdots +f_n$ be the homogeneous decomposition of $f$. Then the homogeneous part of $f^k$ with degree $nk$ is $f^k_n$, so $f^k_n(P)=0$ for all $P \in Z(\mathfrak{a})$. Therefore, $f_n(P)=0$. By induction, we can conclude that $f_i(P)=0$ for all $i$. Hence $f \in I(Z(\mathfrak{a}))$. Conversely, if $f \in I(Z(\mathfrak{a}))$. By homogeneous Nullstellensatz, we have $f_i^{r_i} \in \mathfrak{a}$. Let $r= r_1+\cdots r_n$, then $f^r \in \mathfrak{a}$. Hence $I(Z(\mathfrak{a})) = \sqrt{\mathfrak{a}}$. 

Since $Y \subseteq Z(I(Y))$ and $Z(I(Y))$ is closed, we have $\bar{Y} \subseteq Z(I(Y))$. There is homogeneous ideal $\mathfrak{a}$ such that $\bar{Y} = Z(\mathfrak{a})$. From $Y \subseteq Z(\mathfrak{a})$, we have $I(Y) \subseteq I(Z(\mathfrak{a})) = \sqrt{\mathfrak{a}}$. Hence $\bar{Y}= Z(\mathfrak{a}) = Z(\sqrt{\mathfrak{a}}) \subseteq Z(I(Y))$. We now conclude that $Z(I(Y))= \bar{Y}$.
\qeds
\begin{exer}
	\begin{enumerate}[a)]
	\item There is a $1-1$ inclusion-reversing correspondence between algebraic sets in $\mathbb{P}^n$, and homogeneous radical ideals of $S$ not equal to $S_+$ does not occur in this correspondence, it is sometimes called the irrelevant maximal of $S$.
	\item An algebraic set $Y \subseteq \mathbb{P}^n$ is irreducible if and only if $I(Y)$ is a prime ideal.
	\item Show that $\mathbb{P}^n$ itself is irreducible.
	\end{enumerate}
\end{exer}
\begin{enumerate}[a)]
	\item If $\mathfrak{a}$ is radical homogeneous ideal of $S$ such that $Z(\mathfrak{a}) \neq 0$, then $I(Z(\mathfrak{a})) = \sqrt{\mathfrak{a}} =\mathfrak{a}$. If $Z(\mathfrak{a})=0$, then $I(Z(\mathfrak{a}))= I(\emptyset) = S$. $Z(\mathfrak{a}) =0$ implies $\mathfrak{a}= S$ or $S_+$. By assumption, $S_+$ is not in the correspondence, so $\mathfrak{a}= S$. Hence $I \circ Z$ is identity functor. Similarly, $Z \circ I$ is also identity. With previous exercise, this correspondence is inclusion-reversing.
	\item If $Y$ is irreducible, then for all $x,y \in I(Y)$, we can let $Y_1= Z(x) \cap Y$ and $Y_2 = Z(y) \cap Y$. Since $Y_1 \cup Y_2 = (Z(x) \cap Y) \cup (Z(y)\cap Y)= Z(xy) \cap Y = Y$, $Y_1= Y$ or $Y_2=Y$ by irreducible condition. It implies that $Z(x)=Y$ or $Z(y)=Y$. So $x \in I(Y)$ or $y \in I(Y)$. Conversely, suppose $I(Y)$ is prime. However, for any closed cover $Y_1 \cup Y_2 =Y$, we have $I(Y)= I(Y_1) \cap I(Y_2)$, therefore $I(Y_1)=I(Y)$ or $I(Y_2)=I(Y)$. Hence $Y_1 = Y$ or $Y_2 =Y$ since they are closed.
	\item $\mathbb{P}^n$ is algebraic set corresponding to radical homogeneous ideal $(0)$. It is prime ideal since $k[x_0, x_1 \cdots x_n]$ is integral domain. So $\mathbb{P}^n$ is irreducible from previous statement.
	\end{enumerate}
\qeds
\begin{exer}
	If $Y$ is a projective variety with homogeneous coordinate ring $S(Y)$, show that $\dim S(Y) = \dim Y +1$.
\end{exer}
$Y$ is projective variety, so let $Y= Z(\mathfrak{p}) \subseteq \mathbb{P}^n$ for some prime homogeneous ideal $\mathfrak{p}$. Hence any descending chain of closed subset of $Y$ corresponds to a descending chain of radical homogeneous ideal containing $\mathfrak{p}$ with the same length. However, $S_+$ is prime homogeneous ideal of $S$ which contains any non-zero ideal of $S$ but doesn't correspond to a algebraic set. Hence the $\dim S(Y) > \dim Y$. Since radical ideals contains $\mathfrak{p}$ except $S_+$ also correspond to closed subsets of $Y$, $\dim Y  \geq \dim S(Y)-1$. Hence $\dim S(Y) = \dim Y +1$.\qeds
\section{Week 3}
\begin{exer}
	A regular function on projective variety is continuous map (view $k$ as affine line $\mathbb{A}^1$).
\end{exer}
Let $Y \subseteq P^n$ be a projective variety, then $Y$ can be covered by affine varieties $U_i=Y \cap A^n_i$, where $A^n_i$ are canonical affine coverings of $\mathbb{P}^n$. If $f\colon Y \to k$ be a regular function, then its restrictions $f_{U_i} \colon U_i \to k$ are continuous map, and since $U_i$ are all open in $Y$, $f$ itself is continuous on $Y$. \qeds
\begin{exer}
	Let $\varphi: \mathbb{A}^1 \to C \hookrightarrow \mathbb{A}^2$ be curve defined as $ t \mapsto (t^2,t^3)$. Obviously, $\varphi$ is $1-1$ correspondence. Prove $\varphi$ is not isomorphism between varieties.
\end{exer}
Suppose the coordinate ring of $\mathbb{A}^2_k$ be $k[x,y]$. Then we have coordinate ring $A(C) \cong k[x,y]/(y^2-x^3)$. From definition of $\varphi$, we can write down its pull-back on coordinate rings
\[
\begin{aligned}
\varphi^* \colon A(C) &\to k[t]\\
f& \mapsto f \circ \varphi
\end{aligned}
\]
If $\varphi$ is isomorphism, then $\varphi^*$ is isomorphism between coordinate ring. Therefore it is also bijection between regular functions. Since $t$ is regular function on $\mathbb{A}^1_k$, it must have preimage $f$ such that $\varphi^*(f)=t$. This means that $f(t^2,t^3)=t$. $f$ is regular on $C$, so it is also regular at point $(0,0)$. Nearby $(0,0)$, $f$ can be written as
\[
\frac{\alpha(x,y)}{\beta(x,y)}
\]
where $\alpha ,\beta$ are polynomials in $k[x,y]$ and $\beta(0,0) \neq 0, \frac{\alpha(t^2,t^3)}{\beta(t^2,t^3)} = t$. It is impossible.
Hence we can conclude $\varphi$ can not be an isomorphism otherwise it will induce isomorphism between coordinate ring. \\ \qeds
\begin{exer}
	Let $S= Z(y_0 y_2 - y_1^2)$ be the surface in $\mathbb{P}^2_k$ with the coordinates $(y_0:y_1:y_2)$. Let $\mathbb{P}^1_k$ be projective line with coordinate ring $k[x_0,x_1]$. Consider morphism \[
	\begin{aligned}
	\varphi: \mathbb{P}^1_k &\to S \subset \mathbb{P}^2_k\\
	(x_0:x_1) &\mapsto (x_0^2: x_0x_1: x_1^2)
	\end{aligned}\] and show that $\varphi$ is isomorphism.
\end{exer}
It is well-defined regular morphism since $(x_0^2)(x_1^2) - (x_0 x_1)^2=0$ and with polynomial in each component. We can see that $\varphi$ is bijection and $\varphi^{-1}$ is defined as
\[
\varphi^{-1} \colon (y_0:y_1:y_2) =\begin{cases}
(y_0: y_1)& \text{if } y_0 \neq 0\\
(y_1: y_2)& \text{if } y_2 \neq 0\\
\end{cases} 
\]
It is well-defined regular morphism since if $y_0$ and $y_2$ are neither equal to $0$ then $(y_0: y_1) = (\frac{y_0}{y_1}:1) = (\frac{y_1}{y_2}:1) = (y_1:y_2)$. And we have $\varphi \circ \varphi^{-1} = id_{S}, \varphi^{-1} \circ \varphi = id_{\mathbb{P}^1_k}$.
\begin{exer}
	Let $Y$ be an affine variety in $\mathbb{A}^n_k$. Show that $K(Y) \cong K(\mathcal{O}_{Y,p})$ for all $p \in Y$.
\end{exer}
\begin{exer}
	Prove that for any integer $0 \leq i \leq n$
	\[
	K(\mathbb{P}^n_k) \cong k(x_0/x_i, \cdots, \widehat{x_i/x_i}, \cdots, x_n/x_i)
	\]
\end{exer}
\begin{exer}
	Equation $x_0^2 + x_1^2 + x_2^2=0$ defines a conic $X \hookrightarrow \mathbb{P}_k^2$. Find $t \in K(X)$ such that $K(X) \cong k(t)$ is a transcendental extension of $k$ with degree 1.
\end{exer}
\end{document}
